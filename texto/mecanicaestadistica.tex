% mecanicaestadistica.tex
%
% Copyright (C) 2025 José A. Navarro Ramón <janr.devel@gmail.com>
% Licencia Creative Commons Recognition Share-alike.
% (CC-BY-SA)

\chapter{Mecánica estadística}
La mecánica estadística es una herramienta muy útil en física para estudiar
las propiedades macroscópicas de un sistema formado por un número enorme de
partículas, sin tener en cuenta el estado individual de todas y cada una de
ellas.

Esto se lleva a cabo encontrando la configuración más probable de las
partículas. Así, el estado macroscópico de un sistema se puede a partir del
número de microestados compatibles con ese estado.
Cuanto mayor sea el número de microestados, mayor será la probabilidad de que
se obtenga ese estado macroscópico. El estado macroscópico en el equilibrio
será el que se relacione con el mayor número de microestados compatibles.
De hecho, desde el punto de vista de la mecánica estadística, la entropía está
relacionada con este número de microestados. Así, el estado en equilibrio de un
sistema aislado será el de mayor entropía.

Este capítulo se desarrolla a nivel básico la mecánica estadística
de Maxwell-Boltzmann, la de Bose-Einstein y la de Planck.

\section{Estadística de Maxwell-Boltzmann}
Boltzmann fue un físico austriaco pionero de la mecánica estadística y se
encontró con la incomprensión de muchos físicos de su tiempo, pues el
atomismo no estaba del todo aceptado. Al poco de morir ---comienzos del siglo
XX--- se acumularon más evidencias a favor de los átomos, de modo que esta
quedó completamente aceptada poco después.
%[Wikipedia]

\subsection{Modelo teórico}
En este modelo, se considera un sistema aislado formado por un número enorme
$N$ de partículas distinguibles de cualquier spin y que están lo
suficientemente separadas entre sí como para que la única interacción entre
ellas sea una colisión elástica; por tanto, se desprecia la interacción a
distancia, esto es, la energía potencial de cualquier tipo.
% [Arthur Beiser]

Un ejemplo típico al que se le puede aplicar el modelo es el de un gas formado
por moléculas ---átomos o grupos de átomos neutros---.

Las $N$ partículas, $n_1, n_2, \cdots, n_k$, se distribuyen entre $k$ estados
de energía creciente,  $u_1, u_2, \cdots, u_k$ . Estos estados pueden ser,
bien estados discretos o energías medias de una secuencia creciente de
intervalos continuos. Incluso, como el modelo es clásico, el número de niveles
$k$ puede ser muy elevado, y se podría considerar que estas forman un
continuo (esto lo utilizaremos más tarde).

En todo momento se conserva el número de partículas y la energía.
\begin{equation}
  \sum_{i=1}^{k} n_i = N
\end{equation}
y la energía:
\begin{equation}
  \sum_{i=1}^{k} n_i u_i = U
\end{equation}












%%% Local Variables:
%%% mode: latex
%%% TeX-engine: luatex
%%% TeX-master: "../retazosfisica.tex"
%%% End:

% mecanicaestadistica.tex
%
% Copyright (C) 2025 José A. Navarro Ramón <janr.devel@gmail.com>
% Licencia Creative Commons Recognition Share-alike.
% (CC-BY-SA)

\chapter{Mecánica estadística}
Este capítulo se desarrolla a nivel básico la mecánica estadística
de Maxwell-Boltzmann, la de Bose-Einstein y la de Planck.

\section{Estadística de Maxwell-Boltzmann}
Boltzmann fue un físico austriaco pionero de la mecánica estadística y se
encontró con la incomprensión de muchos físicos de su tiempo, pues el
atomismo no estaba del todo aceptado. Al poco de morir ---comienzos del siglo
XX--- se acumularon más evidencias a favor de los átomos, de modo que esta
quedó completamente aceptada poco después.
%[Wikipedia]

\subsection{Modelo teórico}
En este modelo, se considera un número $N$ enorme de partículas distinguibles
de cualquier spin y que están lo suficientemente separadas entre ellas como
para que la única interacción entre ellas sea la colisión elástica entre ellas,
por tanto, se desprecia la interacción a distancia entre ellas, esto es,
energía potencial de cualquier tipo, eléctrica o gravitatoria.
% [Arthur Beiser]

Un ejemplo típico al que se le puede aplicar el modelo es el de un gas formado
por moléculas ---átomos o grupos de átomos neutros---.

Las $N$ partículas, $n_1, n_2, \cdots, n_k$, se distribuyen entre $k$ estados
de energía creciente,  $u_1, u_2, \cdots, u_k$ . Estos estados pueden ser,
bien estados discretos o energías medias de una secuencia creciente de
intervalos continuos. Incluso, como el modelo es clásico, el número de niveles
$k$ puede ser muy elevado, y se podría considerar que estas forman un
continuo (esto lo utilizaremos posteriormente).









%%% Local Variables:
%%% mode: latex
%%% TeX-engine: luatex
%%% TeX-master: "../retazosfisica.tex"
%%% End:

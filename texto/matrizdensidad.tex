% matrizdensidad.tex
%
% Copyright (C) 2020-2025 José A. Navarro Ramón <janr.devel@gmail.com>
% 1) Código fuente:
% Licencia GNU-2
%
% 2) Texto legible en cualquier formato: pdf, postscript, html, etc.:
% Licencia Creative Commons Recognition-NonCommercial-ShareAlike.
% (CC-BY-NC-SA)
% ----------------------------------------------------------------------------

\chapter{Matriz Densidad}

\section{Operador densidad}
Vamos a utilizar un espacio de Hilbert $\left(\mathbb{C}^2\right)$, de
dimensión 2, por comodidad
\[
  \mathlarger{\mathcal{H} = \mathbb{C}^2}
\]

Lo siguiente es elegir los símbolos de una base. Se podrían elegir algunos
símbolos como $\ket{\varphi_1}$ y $\ket{\varphi_2}$, $\ket{e_1}$ y $\ket{e_2}$,
$\ket{a}$ y $\ket{b}$, etc. Pero vamos a introducir una nomenclatura que es
propia de la información cuántica y que hace referencia al \emph{qbit}.
De esta forma, la base\footnotemark{} la representaríamos como
\footnotetext{La base está formada por dos vectores en un espacio de Hilbert
  de dimensión 2. En general, tendría $n$ vectores.}
\[
  B = \{\ket{0}, \ket{1}\}
\]

El cero y el uno son solo etiquetas. No hay nada profundo en eso, solo inspiran
al bit clásico. Únicamente decir, por completitud, que por \emph{qbit} se
entiende aquel vector que es una combinación lineal de estos dos vectores de la
base
\[
  \bra{\Psi} = a \ket{0} + b \ket{1}
\]
pero no vamos a entrar en esto. Dejamos los \emph{qbits} a un lado.

A partir de ahora solo consideramos tener un espacio de Hilbert de dimensión 2
y estos dos vectores de la base.






%%% Local Variables:
%%% mode: latex
%%% TeX-engine: luatex
%%% TeX-master: "../retazosfisica.tex"
%%% End:

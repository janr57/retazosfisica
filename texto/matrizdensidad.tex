% matrizdensidad.tex
%
% Copyright (C) 2020-2025 José A. Navarro Ramón <janr.devel@gmail.com>
% 1) Código fuente:
% Licencia GNU-2
%
% 2) Texto legible en cualquier formato: pdf, postscript, html, etc.:
% Licencia Creative Commons Recognition-NonCommercial-ShareAlike.
% (CC-BY-NC-SA)
% ----------------------------------------------------------------------------

\chapter{Matriz Densidad}

\section{Introducción}
Vamos a utilizar un espacio de Hilbert $\left(\mathbb{C}^2\right)$, de
dimensión 2, por comodidad
\[
  \mathlarger{\mathcal{H} = \mathbb{C}^2}
\]

Lo siguiente es elegir los símbolos de una base. Se podrían elegir algunos
símbolos como $\ket{\varphi_1}$ y $\ket{\varphi_2}$, $\ket{e_1}$ y $\ket{e_2}$,
$\ket{a}$ y $\ket{b}$, etc. Pero vamos a introducir una nomenclatura que es
propia de la información cuántica y que hace referencia al \emph{qbit}.
De esta forma, la base\footnotemark{} la representaríamos como
\footnotetext{La base está formada por dos vectores en un espacio de Hilbert
  de dimensión 2. En general, tendría $n$ vectores.}
\[
  B = \{\ket{0}, \ket{1}\}
\]

El cero y el uno son solo etiquetas. No hay nada profundo en eso, solo inspiran
al bit clásico. Únicamente decir, por completitud, que por \emph{qbit} se
entiende aquel vector que es una combinación lineal de estos dos vectores de la
base
\[
  \ket{\Psi} = a \ket{0} + b \ket{1}
\]
pero no vamos a entrar en esto. Dejamos los \emph{qbits} a un lado.

A partir de ahora solo consideramos tener un espacio de Hilbert de dimensión 2
y estos dos vectores de la base.

Como es costumbre cuando se trabaja en un espacio de dimensión finita, podemos
utilizar una expresión matricial. Así, podremos representar los vectores de la
base como
\[
  \ket{0} = \begin{pmatrix}1 \\ 0\end{pmatrix}
  \hspace{4em}
  \ket{1} = \begin{pmatrix}0 \\ 1\end{pmatrix}
\]
y sus duales
\[
  \bra{0} = \ket{0}^\dagger
  = \begin{pmatrix}1 \\ 0\end{pmatrix}^\dagger
  = \begin{pmatrix}1 & 0\end{pmatrix}
  \hspace{4em}
  \bra{1} = \ket{1}^\dagger
  = \begin{pmatrix}0 \\ 1\end{pmatrix}^\dagger
  = \begin{pmatrix}0 & 1\end{pmatrix}
\]

\subsection{Operadores autoadjuntos}
En mecánica cuántica, un operador está asociado a un observable si el adjunto
del operador coincide con él mismo, de manera que sus valores propios serían
reales. Un operador así se llama autoadjunto
\[
  \hat{A}^\dagger = \hat{A}
\]

Cuando un operador $\hat{A}$ es autoadjunto, se puede encontrar una nueva base
propia del operador
\[
  \{\ket{a_1}, \ket{a_2}\}
\]
que, por ser base es ortogonal, y por ser autoadjunto, sus valores propios son
reales
\begin{align*}
  &\hat{A} \ket{a_1} = \lambda_1 \ket{a_1}\\
  &\hat{A} \ket{a_2} = \lambda_2 \ket{a_2}
\end{align*}
donde $\lambda_{1}, \lambda_{2} \in \mathbb{R}$.
Recordemos que estamos en un espacio de dimensión 2, pero esto también es
válido para un espacio finito de dimensión superior.
A menos que se diga lo contrario, los operadores que nos encontremos serán
autoadjuntos.

\subsection{Valor esperado de un operador}
Supongamos que un sistema está descrito por el estado $\ket{\Psi}$. Nos
preguntamos acerca del valor esperado de un operador cuando el sistema está en
ese estado.

Si estamos en un mundo de dimensión 2, y la base que utilizamos es
$\{\ket{0}, \ket{1}\}$, el estado se puede expresar como una combinación lineal
de los vectores de la base
\[
  \ket{\Psi} = a_0 \ket{0} + a_1 \ket{1}
  \text{, donde }
  a_i \in \mathbb{C}
\]

Se define el \emph{valor esperado del observable} $\hat{A}$, cuando el sistema
está en el estado $\ket{\Psi}$, mediante el producto escalar
\begin{align*}
  \braket{\hat{A}}_{\Psi}
  &= \braket{\Psi | \hat{A} | \Psi}
  = \left(a_{0}^{*} \bra{0} + a_{1}^{*} \bra{1}\right)
  \hat{A}
    \left(a_{0} \ket{0} + a_{1} \ket{1}\right)\\
  &= a_{0}^{*} a_{0} \braket{0|\hat{A}|0}
    + a_{0}^{*} a_{1} \braket{0|\hat{A}|1}
    + a_{1}^{*} a_{0} \braket{1|\hat{A}|0}
    + a_{1}^{*} a_{1} \braket{1|\hat{A}|1}\\
  &= a_{0} a_{0}^{*} \braket{0|\hat{A}|0}
    + a_{1} a_{0}^{*} \braket{0|\hat{A}|1}
    + a_{0} a_{1}^{*} \braket{1|\hat{A}|0}
    + a_{1} a_{1}^{*} \braket{1|\hat{A}|1}\\
  &= a_{0} a_{0}^{*} A_{00}
    + a_{1} a_{0}^{*} A_{01}
    + a_{0} a_{1}^{*} A_{10}
    + a_{1} a_{1}^{*} A_{11}\\
  &= \rho_{00} A_{00}
    + \rho_{10} A_{01}
    + \rho_{01} A_{10}
    + \rho_{11} A_{11}
\end{align*}
donde hemos utilizado $A_{ij} = \braket{i|\hat{A}|j}$.

Si ahora llamamos $\rho_{ij} = a_{i} a_{j}^{*}$, obtenemos
\begin{equation}
  \braket{\hat{A}}_{\Psi}
  = \braket{\Psi | \hat{A} | \Psi}
  = \rho_{00} A_{00}
    + \rho_{10} A_{01}
    + \rho_{01} A_{10}
    + \rho_{11} A_{11}  
\end{equation}
Obsérvese que el valor esperado de un operador autoadjunto es un número real.

Las $\rho_{ij}$ las podríamos interpretar como elementos de una matriz $2\times 2$.
Por otro lado, sabemos por mecánica cuántica que las $A_{ij}$ son elemenntos de una
matriz. Así que
\[
  \mmm{\rho}
  = \begin{pmatrix}\rho_{00} & \rho_{01} \\ \rho_{10} & \rho_{11} \end{pmatrix}
  ;\hspace{4em}
  \mmm{A}
  = \begin{pmatrix}A_{00} & A_{01} \\ A_{10} & A_{11} \end{pmatrix}
\]

Pero, al multiplicar estas dos matrices $2\times 2$ obtenemos otra matriz
$2\times 2$, no un número, aunque, si nos fijamos en los elementos diagonales
\[
  \mmm{\rho} \mmm{A}
  = \begin{pmatrix}\rho_{00} & \rho_{01} \\ \rho_{10} & \rho_{11} \end{pmatrix}
  \begin{pmatrix}A_{00} & A_{01} \\ A_{10} & A_{11} \end{pmatrix}
  = \begin{pmatrix}
    \rho_{00} A_{00} + \rho_{01} A_{10} & \cdots \\
    \cdots & \rho_{10} A_{01} + \rho_{11} A_{11}
  \end{pmatrix}
\]

Pero, si nos fijamos en los elementos diagonales, observamos que el valor
esperado es la suma de estos. La suma de los elementos diagonales de una matriz
cuadrada es la \emph{traza}
\[
  \braket{\hat{A}}_{\Psi}
  = \braket{\Psi|\hat{A}|\Psi}
  = \text{Tr}(\mmm{\rho}\mmm{A})
\]

¿Por qué nos complicamos tanto para calcular el valor esperado de un observable
en el estado $\ket{\Psi}$, si lo podemos calcular directamente como
$\braket{\Psi|\hat{A}|\Psi}$?

Bueno, sí y no. Resulta que en estadística, o incluso, en teoría de información
cuántica, puede ocurrir que nos digan que la probabilidad de que un sistema se
encuentre en un estado, digamos, $\ket{\alpha}$, sea $1/3$ y que la de
encontrarse en $\ket{\beta}$ fuera $2/3$. Es decir, que no nos aseguren que el
sistema se encuentre en un estado cuántico concreto.

Este tipo de situación introduce una incertidumbre clásica, pues estas
probabilidades son absolutamente clásicas. Además, en los estados cuánticos
$\ket{\alpha}$ y $\ket{\beta}$ también hay probabilidades cuánticas.

Todo esto nos complica bastante la vida, y su desarrollo sería muy laborioso,
muy difuso y muy poco claro, si utilizamos la nomenclatura estándar del cálculo
vectorial de toda la vida, y resulta que introduciendo esta matriz densidad
$\mmm{\rho}$, todo es mucho más fácil e intuitivo.

Podemos observar que la matriz densidad depende solo de los coeficientes de la
función de onda en nuestra base, $\ket{\Psi} = a_0 \ket{0} + a_1 \ket{1}$
\[
  \mmm{\rho]}
  = \begin{pmatrix}\rho_{00} & \rho_{01}\\ \rho_{10} & \rho_{11}\end{pmatrix}
  = \begin{pmatrix}
    a_{0} a_{0}^{*} & a_{0} a_{1}^{*} \\
    a_{1} a_{0}^{*} & a_{1} a_{1}^{*}
    \end{pmatrix}
\]


\subsection{Estado puro}
Ahora mostraremos otra forma de obtener la matriz densidad
\[
  \ket{\Psi}\bra{\Psi}
  = \begin{pmatrix}a_0 \\ a_1\end{pmatrix}
  = \begin{pmatrix}a_{0}^{*} & a_{1}^{*}\end{pmatrix}
  = \begin{pmatrix}
    a_{0} a_{0}^{*} & a_{0} a_{1}^{*} \\
    a_{1} a_{0}^{*} & a_{1} a_{1}^{*}
  \end{pmatrix}
  = \mmm{\rho}
\]

Decimos que un sistema se encuentra en un estado cuántico puro si
\begin{equation}
  \mmm{\rho} = \ket{\Psi} \bra{\Psi}
\end{equation}

\section{Propiedades de la matriz densidad}





%%% Local Variables:
%%% mode: latex
%%% TeX-engine: luatex
%%% TeX-master: "../retazosfisica.tex"
%%% End:
